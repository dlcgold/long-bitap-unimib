\documentclass[a4paper,12pt, oneside]{article}
% \usepackage{fullpage}
\usepackage[italian]{babel}
\usepackage[utf8]{inputenc}
\usepackage{amssymb}
\usepackage{amsthm}
\usepackage{graphics}
\usepackage{amsfonts}
\usepackage{listings}
\usepackage{amsmath}
\usepackage{amstext}
\usepackage{engrec}
\usepackage{rotating}
\usepackage[safe,extra]{tipa}
\usepackage{showkeys}
\usepackage{multirow}
\usepackage{hyperref}
\usepackage{sectsty}
\usepackage{mathtools}
\usepackage{microtype}
\usepackage{enumerate}
\usepackage{braket}
\usepackage{marginnote}
\usepackage{pgfplots}
\usepackage{cancel}
\usepackage{polynom}
\usepackage{booktabs}
\usepackage{enumitem}
\usepackage{framed}
\usepackage{algorithm}
\usepackage{algpseudocode}
\usepackage{pdfpages}
\usepackage{pgfplots}
\usepackage[cache=false]{minted}


\title{Relazione Progetto\\
  Elementi di Bioinformatica\\
  \large Long Bitap}

\author{Davide Cozzi\\
  829827\\
  \href{mailto:d.cozzi@campus.unimib.it}{d.cozzi@campus.unimib.it}}
\date{}

\pgfplotsset{compat=1.13}
\begin{document}
\maketitle

\definecolor{shadecolor}{gray}{0.80}
\setlist{leftmargin = 2cm}
\newtheorem{teorema}{Teorema}
\newtheorem{definizione}{Definizione}
\newtheorem{esempio}{Esempio}
\newtheorem{corollario}{Corollario}
\newtheorem{lemma}{Lemma}
\newtheorem{osservazione}{Osservazione}
\newtheorem{nota}{Nota}
\newtheorem{esercizio}{Esercizio}

\renewcommand{\chaptermark}[1]{%
  \markboth{\chaptername
    \ \thechapter.\ #1}{}}
\renewcommand{\sectionmark}[1]{\markright{\thesection.\ #1}}
\allsectionsfont{\centering}

\section*{Bitap}
\textit{Nell'analisi un bit rappresentato in colonna presenta il bit
  più significativo, \textit{MSB}, in basso}\\
Innanzitutto si descrive l'algoritmo di base, funzionante per pattern
lunghi al massimo quanto una \textit{word}, \texttt{w} della cpu.\\
Si hanno in input un \textit{pattern} \texttt{P} di lunghezza \texttt{p} e un
\textit{testo} \texttt{T} di lunghezza \texttt{t}. Si ha quindi \texttt{p <= w}.\\
A livello teorico si costruisce una \textbf{matrice booleana}
\texttt{D}, di dimensioni $\mathtt{p}\times \mathtt{t}$, e si stabiliscono due indici:
\begin{enumerate}
  \item \texttt{i} che itera sul pattern
  \item \texttt{j} che itera sul testo
\end{enumerate}
Si ha che la generica posizione di indici \texttt{i}, \texttt{j} nella
matrice è $1$ sse i primi \texttt{i} caratteri del pattern matchano un
numero \texttt{i} di caratteri del testo terminanti all'indice
\texttt{j}.\\
Si ha quindi la seguente \textbf{equazione di ricorrenza}:
\[D[i,j]=
  \begin{cases}
    1 & \mbox{sse } \mathtt{P}[1..\mathtt{i}] =
    \mathtt{T}[\mathtt{j}-\mathtt{i}+1..\mathtt{j}] \\
    0 & \mbox{altrimenti}
  \end{cases}
\]
Nell'\texttt{i}-sima riga si ha che le occorrenze di 1 indicano i
punti nel testo dove termina una copia di
$\mathtt{P}[1..\mathtt{i}]$.\\
Invece la \texttt{j}-sima colonna mostra tutti i prefissi del testo
che finiscono nella posizione \texttt{j} del testo.\\
Nell'ultima riga della matrice si ha la soluzione, ovvero si ha 1 dove
termina un match del pattern nel testo.\\
In termini pratici questo algoritmo può essere costruito mediante
operazioni \textbf{bit a bit} in quanto le singole colonne della
matrice teorica possono essere viste come numeri in rappresentazione
binaria.\\
Per procedere si ha una fase di preprocessamento in cui si costruisce
un array \texttt{U}, di lunghezza pari a quella dell'alfabeto in uso,
che contiene in posizione \texttt{k} il binario rappresentante le
occorrenze del carattere \texttt{k} nel pattern. Nel pattern il
vettore \texttt{U} viene definito di lunghezza \texttt{CHAR\_MAX},
ovvero $127$, per usare la tabella ASCII a 7 bit. Per costruire tale
array si procede inizializzando tutte le celle a 0. Si itera poi lungo
il pattern aggiornando \texttt{U} nella posizione del carattere preso
in considerazione del pattern facendo l'\texttt{or} tra l'attuale
contenuto di \texttt{U} in quella posizione e il numero la cui
rappresentazione binaria presenta 1 solo nella posizione di indice
\texttt{i} (questo comportamento è ottenibile con il
\textit{left-shift} di 1 di \textit{posizioni}). Dopo aver iterato su
tutto il pattern ottengo il vettore \texttt{U} correttamente caricato.
\begin{esempio}
  Immaginiamo un pattern semplice: \textit{caac}.\\
  Il primo carattere è \texttt{c} e \texttt{i} è in posizione
  0. \texttt{U} è ancora caricato con soli zeri. Quindi si ha:
  \[
    U[\mathtt{c}]=U[99]=
    \begin{matrix}
      0\\
      0\\
      0\\
      0\\
    \end{matrix}\vee
    \begin{matrix}
      1\\
      0\\
      0\\
      0\\
    \end{matrix}=
    \begin{matrix}
      1\\
      0\\
      0\\
      0\\
    \end{matrix}
  \]
  proseguo con \texttt{a} e shifto \texttt{i} di 1:
  \[
    U[\mathtt{a}]=U[97]=
    \begin{matrix}
      0\\
      0\\
      0\\
      0\\
    \end{matrix}\vee
    \begin{matrix}
      0\\
      1\\
      0\\
      0\\
    \end{matrix}=
    \begin{matrix}
      0\\
      1\\
      0\\
      0\\
    \end{matrix}
  \]
  il carattere successivo è ancora \texttt{a}:
  \[
    U[\mathtt{a}]=U[97]=
    \begin{matrix}
      0\\
      1\\
      0\\
      0\\
    \end{matrix}\vee
    \begin{matrix}
      0\\
      0\\
      1\\
      0\\
    \end{matrix}=
    \begin{matrix}
      0\\
      1\\
      1\\
      0\\
    \end{matrix}
  \]
  e infine trovo ancora \texttt{c}:
  \[
    U[\mathtt{c}]=U[99]=
    \begin{matrix}
      1\\
      0\\
      0\\
      0\\
    \end{matrix}\vee
    \begin{matrix}
      0\\
      0\\
      0\\
      1\\
    \end{matrix}=
    \begin{matrix}
      1\\
      0\\
      0\\
      1\\
    \end{matrix}
  \]
  Alla fine quindi $U[\mathtt{c}]=1001|_2=9$ e $U[\mathtt{a}]=0110|_2=6$
\end{esempio}
L'algoritmo prosegue inizializzando la prima colonna a 1 se testo e
pattern condividono il primo carattere, 0 altrimenti.\\
Si procede poi col calcolo delle colonne successive alla prima
sfruttando la colonna precedente. Si procede con un
\textit{left-shift} del valore rappresentante la colonna precedente
con l'aggiunta di un 1 in testa. Si procede poi con l'\texttt{and} tra
il risultato appena ottenuto e il valore di \texttt{U} nella posizione
del carattere che sto considerando. Si controlla infine ogni valore
rappresentante una colonna vedendo se presenta 1 nell'ultimo bit. Per
ottenere questo risultato si procede con il \textit{left-shift} di uno
di un'unità pari alla lunghezza del pattern meno uno e
all'\texttt{and} con il valore rappresentante la colonna. Mi verrà
infatti restituito un binario avente valore o 0 o $2^{p-1}$ (nel caso di
word grandi 3 avrei $100|_2=4$ avendo quindi 1 nell'ipotetica ultima
riga della matrice, avendo quindi un match, infatti viene fatto un
\texttt{and} con un valore che in codifica binaria presenta tutti 0
tranne il \textit{MSB} che presenta 1).\\
Il limite di queto algoritmo è hardware e consiste nella
rappresentazione (e quindi anche nelle operazioni) su binari oltre il
numero di bit della word.\\
Con questa soluzione si ha un tempo di:
\[\sim O(p + t)\]
\section*{Slow Long Bitap}
\textit{Questa è la mia implementazione più naive per superare il limite della
  grandezza della word}.\\
Procedo innanzitutto separando il pattern in \textit{sottopatterns}
lunghi \texttt{w}, tranne l'ultimo, che sarà lungo $\mathtt{p} -
\mathtt{w}$. Spezzo quindi il pattern in un numero di sottopatterns
pari a $\big\lceil \frac{\mathtt{p}}{\mathtt{w}}\big\rceil$. Questi patterns
vengono caricati in un array di stringhe.\\
Per effettuare il match si sfruttano 3 array di lunghezza
\texttt{t}. Si ha un array contenente l'ultima riga teorica prodotta
dall'algoritmo \texttt{bitap}, uno contenente quello precedente e uno
risultante. Si procede a coppie verificando che l'array dell'ultimo
bitap presenti 1 esattamente ad una distanza pari alla lunghezza del
sottopattern in analisi rispetto ad un 1 nell'array prodotto
dall'algoritmo \texttt{bitap} sul sottopattern precedente. Nel caso si
abbia questa corrispondenza tra i due array si carica 1 nel terzo array (quello
risultante) in corrispondenza dell'indice in cui c'era 1 nell'array
prodotto per il sottopattern corrente, altrimenti si carica 0. Alla
successiva iterazione l'array risultante diventerà quello precedente
fino ad esaurimento dei sottopatterns. A questo punto avrò un array
risultante con gli indici che rappresentano la fine di un match di
\texttt{P} in \texttt{T}. Per ottenere le posizioni di inizio basta
sottrarre a tali valori $\mathtt{p} - 1$ (il $-1$ è causato
dall'inizio all'indice 0 e non 1).
\begin{esempio}
  Vediamo un esempio semplificato ipotizzando una cpu con
  $WORDSIZE=2$.\\
  Sia $\mathtt{T}=abbaccabbacabcabbacc$ e $\mathtt{P}=abbac$.\\
  Si avrà quindi il seguente array dei sottopatterns:
  \[\mathtt{patterns}=\{"ab", "ba", "c"\}\]
  Per semplicità rappresentiamo la tabella complessiva dei tre
  \texttt{bitap} senza rappresentare i vari steps intermedi. Si
  otterrebbe quindi la matrice dei 3 bitap (dove vengono sottolineate
  le occorrenze di uno valide secondo la logica sopra descritta):
  \begin{center}
    \begin{tabular}{|c|c|c|c|c|c|c|c|c|c|c|c|c|c|c|c|c|c|c|c|c|}
      \hline
      & 0 & 1 & 2 & 3 & 4 & 5 & 6 & 7 & 8 & 9 & 10 & 11 & 12 & 13 &
                                                                    14
      & 15 & 16 & 17 & 18 & 19 \\ \hline
      & a & b & b & a & c & c & a & b & b & a & c & a & b & c & a &
                                                                    b
           & b & a & c & c \\ \hline\hline  
      \textit{ab} & 0 & \underline{1} & 0 & 0 & 0 & 0 & 0 &
                                                            \underline{1}
                                      & 0
                                          & 
                                            0
                                              & 0 & 0 & 1 & 0 & 0 &
                                                                    \underline{1} 
           & 0 & 0 & 0 & 0 \\ \hline   
      \textit{ba} & 0 & 0 & 0 & \underline{1} & 0 & 0 & 0 & 0 & 0 &
                                                                    \underline{1}
                                                                    
                                              & 0 & 0 & 0 & 0 & 0 &
                                                                    0
           & 0 & \underline{1} & 0 & 0 \\ \hline   
      \textit{c} & 0 & 0 & 0 & 0 & \underline{1} & 1 & 0 & 0 & 0 & 0
                                              & \underline{1} & 0 &
                                                                    0 & 1 & 0 & 
                                                                                0
           & 0 & 0 & \underline{1} & 1 \\ \hline
    \end{tabular}

  \end{center}
  Si ha quindi un match terminante all'indice 4 (quindi iniziante all'indice
  0), uno all'indice 10 (quindi iniziante all'indice 6)
  e uno all'indice 18 (quindi iniziante all'indice 14).
\end{esempio}.
\\
Visto che viene effettuata una chiamata a \texttt{bitap} per ogni
sottopattern si ha un tempo pari a (ipotizzando un tempo $O(m)$, con
$m\sim t$, per la \texttt{memcpy}):
\[\sim O\left(\Big\lceil
    \frac{\mathtt{p}}{\mathtt{w}}\Big\rceil\cdot (m + w + 2\cdot
    t)\right)\to\,\sim 
  O\left(\Big\lceil \frac{\mathtt{p}}{\mathtt{w}}\Big\rceil\cdot (m + w
    + t)\right)\] 
Si notano quindi le problematiche che posso nascere all'aumentare
della grandezza del testo e a quella del pattern.\\
Il problema grave di questa implementazione è che vengono effettuate
troppe operazioni inutili a priori chiamando bitap su tutto il testo
per tutti i sottopattern.\\
In termini pratici (ma meno oggettivi) sulla mia macchina
(\textit{Intel i7 8550U} con $8$gb di ram) per un pattern di $3374$
caratteri su un testo lungo $3391270$ si ha un tempo di circa $2.2$
secondi per trovare le $60$ occorrenze.
\newpage
\section*{Fast Long Bitap}
Il ragionamento di base è lo stesso della precedente implementazione e
si procede spezzando il pattern in maniera analoga. La differenza
sostanziale si ritrova nella finestra di testo su cui viene chiamata
la funzione \texttt{bitap} e il numero di volte in cui essa viene
chiamata.\\
Innanzitutto viene chiamata la funzione \texttt{bitap} sul primo
sottopattern (qualora si abbia un solo sottopattern si è nella
situazione in cui $\mathtt{p}<=\mathtt{w}$ e quindi si stampano
direttamente le occorrenze). Si chiama anche una funzione
\texttt{countfirst} che, preso in ingresso un array, restituisce un
secondo array contenente gli indici dove si ha un 1. Chiamando questa
funzione sull'array restituito dalla funzione \texttt{bitap}, chiamata
sul primo sottopattern, calcolo gli indici in cui termina un match del
primo sottopattern nel testo. Questo passaggio è \textbf{fondamentale}
per ridurre il numero di operazioni inutili in quanto a priori gli
eventuali match dell'intero pattern avranno come indice di partenza
gli indici ricavabili come sopra a partire da questi ultimi.\\
Procedendo col ragionamento si ha che si può ricercare il secondo
sottopattern solo nella finestra di testo, che parte con l'indice di
fine match del primo sottopattern, di lunghezza pari a questo secondo
sottopattern. A questo punto chiamo quindi \texttt{bitap} su questa
piccola finestra di testo e ne valuto solo l'ultimo elemento. Si hanno
quindi 3 casi:
\begin{enumerate}
  \item l'ultimo elemento è 1 e ho un altro sottopattern da
  analizzare. Proseguo quindi iterativamente fino ad esaurimento dei
  sottopatterns
  \item l'ultimo elemento è 1 e non ho un altro sottopattern da
  analizzare. In tal caso ho analizzato l'intero pattern e nell'ultimo
  indice raggiunto termina un match del pattern sul testo, che viene
  stampato. A questo punto vedo se esiste un ulteriore valore
  nell'array contenente gli indice di fine match del primo
  sottopattern. Se esiste ricomincio da capo a partire da quell'indice
  altrimenti termino l'esecuzione in quanto ho concluso l'analisi
  \item l'ultimo elemento è 0. In tal caso a priori smetto di cercare
  un pattern iniziante da un dato valore dell'array contenente gli
  indice di fine match del primo sottopattern ed eventualmente mi
  sposto al valore successivo ricominciando da capo la ricerca a
  partire dal nuovo indice. Qualora tale valore non esista  termino
  l'esecuzione in quanto ho concluso l'analisi 
\end{enumerate}
Si nota che le chiamate a \texttt{bitap} vengono effettuate su un
pattern e un testo di uguale lunghezza.
\begin{esempio}
  Vediamo un esempio semplificato ipotizzando una cpu con
  $WORDSIZE=2$.\\
  Sia $\mathtt{T}=abbaccabbacabcabbacc$ e $\mathtt{P}=abbac$.\\
  Si avrà quindi il seguente array dei sottopatterns:
  \[\mathtt{patterns}=\{"ab", "ba", "c"\}\]
  Chiamo \texttt{bitap} sul primo sottopattern, ovvero
  ``\textit{ab}'':
  \begin{center}
    \begin{tabular}{|c|c|c|c|c|c|c|c|c|c|c|c|c|c|c|c|c|c|c|c|c|}
      \hline
      & 0 & 1 & 2 & 3 & 4 & 5 & 6 & 7 & 8 & 9 & 10 & 11 & 12 & 13 &
                                                                    14
      & 15 & 16 & 17 & 18 & 19 \\ \hline
      & a & b & b & a & c & c & a & b & b & a & c & a & b & c & a & b &
                                                                        b
                & a & c & c \\ \hline  
      \textit{ab} & 0 & 1 & 0 & 0 & 0 & 0 & 0 & 1 & 0 & 0 & 0 & 0 & 1 & 0 & 0 & 1 &
                                                                                    0
                & 0 & 0 & 0 \\ \hline
    \end{tabular}
  \end{center}
  l'array calcolato con \texttt{countfirst} conterrà i valori:
  \[\{1,7,12,15\}\]
  Parto quindi dall'indice 1. Chiamo bitap sul secondo sottopattern,
  ``\textit{ab}'', lungo 2, nella porzione di testo tra 2 e
  3 (inclusi). La chiamata \texttt{bitap} di ``\textit{ba}'' su ``\textit{ba}''
  restituisce $\{0,1\}$. Avendo 1 in ultima porzione posso proseguire.
  Chiamo \texttt{bitap} il terzo sottopattern, ``\textit{c}'', lungo
  1, sulla finestra di testo che parte da 4 e termina in 4 (inclusi). Ovviamente
  essendo ``\textit{c}'' uguale a ``\textit{c}'' ottengo l'array con
  solo un elemento: $\{1\}$. Avendo finito i sottopattern e avendo 1
  alla fine so che ho un match terminante in posizione 4 (e quindi
  iniziante in posizione 0). \\Mi sposto quindi sul secondo indice di
  match del primo sottopattern: 7. Procedendo come sopra scopro che
  anche in questo caso ho un match, terminante all'indice 10 (e quindi
  che inizia all'indice 6). \\Valuto quindi 12.  Chiamo bitap sul
  secondo sottopattern, ``\textit{ab}'', lungo 2, nella porzione di
  testo tra 2 e 3 (inclusi). La chiamata \texttt{bitap} di
  ``\textit{ba}'' su ``\textit{ca}''  restituisce $\{0,0\}$. Smetto
  quindi di analizzare altro e passo a valutare 15. Anche in questo
  caso si avrà un match, terminante in 18 e iniziante in 14.\\
  Ho quindi trovato gli indici di partenza dei 3 match:
  \[\{0,6,14\}\]
\end{esempio}
Diamo ora una stima approssimata dei tempi. Chiamo $k$ il numero di
occorrenze del primo pattern nel testo. Sapendo che il costo del
calcolo di una sottostringa ha un costo pari alla lunghezza della
sottostringa (e sapendo che in generale queste saranno lunghe
\texttt{w}) si ha che ogni chiamata alla funzione che calcola la
sottostringa costa $\sim O(\mathtt{w})$. Come detto la funzione
\texttt{bitap} viene chiamata su un testo lungo quanto il sottopattern
(che sappiamo essere lungo al più \texttt{w}), 
quindi il suo costo è $\sim O(2\cdot w)\to\sim O(w)$. Quindi il costo
dell'algortimo è di (imponendo che $n$ rappresenta il numero di
sottopatterns, ovvero $n=\big\lceil
  \frac{\mathtt{p}}{\mathtt{w}}\big\rceil$):
\[\sim O(k\cdot n\cdot (2w))
  \to\sim O(k\cdot n\cdot w)
\]
a questa stima vanno sommati i costi della prima chiamata a
\texttt{bitap}, ovvero $\sim O(w + t)$ e del calcolo deglin indici di
partenza, ovvero $\sim O(t)$. Si nota come non si abbia più \texttt{t}
moltiplicato per il numero di sottopatterns (fattore che rallentava
l'algoritmo in presenza di testi molto lunghi).
\newpage
\section*{Tabella Test}
In termini pratici (ma meno oggettivi) sulla mia macchina
(\textit{Intel i7 8550U} con $8$gb di ram) sono stati effettuati 4
test:
\begin{enumerate}
  \item \textbf{short}: un breve pattern (di grandezza minore a word)
  su un breve testo:
  \begin{itemize}
    \item lunghezza pattern: $7$
    \item lunghezza testo: $26$
    \item numero di occorrenze: $2$
  \end{itemize}
  \item \textbf{word}: un breve pattern lungo quanto una word su un
  testo non breve:
  \begin{itemize}
    \item lunghezza pattern: $64$
    \item lunghezza testo: $74640$
    \item numero di occorrenze: $289$
  \end{itemize}
  \item \textbf{medium}: un pattern lungo circa il triplo di una word
  su un testo non breve:
  \begin{itemize}
    \item lunghezza pattern: $151$
    \item lunghezza testo: $37287$
    \item numero di occorrenze: $170$
  \end{itemize}
  \item \textbf{long}: un pattern lungo su un testo molto lungo:
  \begin{itemize}
    \item lunghezza pattern: $3374$
    \item lunghezza testo: $3391270$
    \item numero di occorrenze: $60$
  \end{itemize}
\end{enumerate}
Vediamo quindi una tabella riassuntiva coi tempi calcolati da \texttt{time}:
\begin{center}
  \begin{tabular}{|c|c|c|c|c|}
    \hline
    & short        & word         & medium       & long        \\ \hline\hline
    slow & $\sim 0.001s$ & $\sim 0.004s$ & $\sim 0.005s$ & $\sim 1.6s$  \\ \hline
    fast & $\sim 0.001s$ & $\sim 0.004s$ & $\sim 0.004s$ & $\sim 0.06s$ \\ \hline
  \end{tabular}
\end{center}
\textbf{Questi dati non sono universalmente utili ma sono
  identificativi del miglioramento delle performance in presenza di
  pattern e testo lunghi}
\end{document}
